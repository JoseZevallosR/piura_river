\documentclass{beamer}
\usepackage[utf8]{inputenc}
\usepackage{graphicx}
\usepackage{lipsum}
\usepackage{tikz}

\usetheme{Madrid}


% Logo solo en la diapositiva de título
\titlegraphic{
  \includegraphics[width=2.5cm]{64654f89-9b8a-47a1-9e96-0b299dc2834c.png}
}

% Datos de presentación
\title[Escenarios de riesgo ante precipitaciones]{ESCENARIOS DE RIESGO ANTE PRECIPITACIONES:\\Evaluación de fenómenos hidrometeorológicos futuros en zonas vulnerables a través de modelos hidrológicos replicables}
\author{
  Luis Izquierdo Horna \\ \texttt{lizquierdo@utp.edu.pe} \and
  Thurian Leonel Cevallos Vivar \\ \texttt{U17306177@utp.edu.pe} \and
  Jose Augusto Zevallos Ruiz \\ \texttt{C19819@utp.edu.pe}
}
\institute{Universidad Tecnológica del Perú (UTP)}
\date{Mayo 2025}

\begin{document}

\begin{frame}
  \titlepage
\end{frame}

\begin{frame}{Contenido}
  \tableofcontents
\end{frame}

\section{Introducción}
\begin{frame}{Introducción}
  \small
  \begin{itemize}
    \item Breve descripción del problema de riesgo ante precipitaciones extremas.
    \item Importancia de modelos hidrológicos en la planificación preventiva.
    \item Objetivo: evaluar escenarios futuros en zonas vulnerables.
  \end{itemize}
\end{frame}

\section{Metodología}
\begin{frame}{Metodología}
  \small
  \begin{itemize}
    \item Modelos hidrológicos replicables basados en datos históricos.
    \item Aplicación de escenarios climáticos proyectados (p.ej., RCP 4.5 y 8.5).
    \item Validación con eventos extremos pasados.
  \end{itemize}
\end{frame}

\section{Resultados esperados}
\begin{frame}{Resultados esperados}
  \small
  \begin{itemize}
    \item Mapas de riesgo por cuenca.
    \item Modelos replicables para planificación local.
    \item Recomendaciones para estrategias de adaptación.
  \end{itemize}
\end{frame}

\section{Conclusiones}
\begin{frame}{Conclusiones}
  \small
  \begin{itemize}
    \item La replicabilidad de los modelos permite su uso en múltiples zonas vulnerables.
    \item Se promueve una gestión del riesgo basada en evidencia y proyecciones climáticas.
  \end{itemize}
\end{frame}

\begin{frame}{Gracias}
  \centering
  \Large ¿Preguntas?
\end{frame}

\end{document}
